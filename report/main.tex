\documentclass[12pt,a4paper]{report}

\usepackage[italian]{babel}
\usepackage{newlfont}
\usepackage{graphicx}
\usepackage{version}

\textwidth=450pt\oddsidemargin=0pt

\begin{document}

\begin{titlepage}
\begin{center}
{{\Large{\textsc{Alma Mater Studiorum $\cdot$ Universit\`a di
Bologna}}}} \rule[0.1cm]{15.8cm}{0.1mm}
\rule[0.5cm]{15.8cm}{0.6mm}
{\small{\bf DIPARTIMENTO DI INFORMATICA – SCIENZA E INGEGNERIA\\
Corso di Laurea in Ingegneria e Scienze Informatiche }}
\end{center}
\vspace{15mm}
\begin{center}
{\LARGE{\bf PARALLELIZZAZIONE SU GPU}}\\
\vspace{3mm}
{\LARGE{\bf ALGORITMO MARCHING SQUARES}}\\
\vspace{3mm}
{\LARGE{\bf PER APPLICAZIONE INDUSTRIALE}}\\
\vspace{25mm}

Elaborato in:\\
High Performance Computing\\
\end{center}
\vspace{25mm}
\par
\noindent
\begin{minipage}[t]{0.47\textwidth}
{\large{\bf Relatore:\\
Chiar.mo Prof.\\
Moreno Marzolla\\}}
{\large{\bf \\Correlatore:\\
Dott.\\
Matteo Roffilli}}
\end{minipage}
\hfill
\begin{minipage}[t]{0.47\textwidth}\raggedleft
{\large{\bf Presentata da:\\
Alessandro Sciarrillo}}
\end{minipage}
\vspace{30mm}
\begin{center}
{\large{\bf 
%Sessione\\%inserire il numero della sessione in cui ci si laurea
Anno Accademico 2022/2023}}%inserire l'anno accademico a cui si Ë iscritti
\end{center}
\end{titlepage}

\begin{comment}

\documentclass[a4paper,12pt]{report}
\AddToHook{cmd/tableofcontents/after}{\clearpage}

\usepackage{graphicx} % Required for inserting images
\usepackage{newtxtext}
\usepackage{hyperref}
\usepackage{tabularx}
\usepackage[english,italian]{babel}

\date{}
% \begin{document}
{
    \begin{center}
        ALMA MATER STUDIORUM - UNIVERSITA' DI BOLOGNA\\
        CAMPUS DI CESENA\\[10pt]
    
        %\hline \\[10pt]

        \par\noindent\rule{\textwidth}{0.4pt}
        \par\noindent\rule{\textwidth}{1.6pt} \\[10pt]
        
        DIPARTIMENTO DI INFORMATICA – SCIENZA E INGEGNERIA\\[20pt]
    
        Corso di Laurea in Ingegneria e Scienze Informatiche\\[80pt]
        
        \textbf{PARALLELIZZAZIONE SU GPU DELL'ALGORITMO MARCHING SQUARES PER APPLICAZIONE INDUSTRIALE}\\[80pt]
        
        Elaborato in:\\
        High Performance Computing\\[50pt]
    \end{center}
    \textbf{Relatore: \hfill Presentata da: \\
    Prof. Moreno Marzolla \hfill Alessandro Sciarrillo\\[5pt]
    \raggedright{Correlatore:\\Dott. Matteo Roffilli}\\[50pt]}
    
    
    \begin{center}
        Anno Accademico 2022/2023
    \end{center}
    \hypersetup{linkcolor=black}
    \tableofcontents
}
\end{comment}

\chapter{Abstract}
% una vista rapida della tesi 1/2 facciate, cosa si andrà a fare, quale è il problema, cosa si fa per risolverlo e quali risultati si otterranno. 
% keyword: cosa siANDRà a fare, OTTERREMO

\chapter{Introduzione}
% spiegare il problema e la sua utilità, obbiettivo

\chapter{Introduzione alla programmazione Parallela}
% guardare altre tesi per vedere in che ordine metterlo

\chapter{Tentativi Svolti}
% valutare cambio nome
% spiegare tutti i tentativi svolti, 1 o più capitoli per ogni veersione o test


\chapter{Risultati Ottenuti}
% speedup, ecc.. In sostanza si descrivono e commentano i grafici ottenuti
% se si vuole misurare tempo di 2° paarte di find_contours() poi meglio lasciare distinte le due misurazioni (lib completa e assemble contours) in modo di dare a chi legge la possibilità di valutare la precisione non perfetta della misura

\chapter{Conclusioni}
% speculare all'abstract, si dicono le stesse cose ma invece che "andremo a fare"  si dirà "abbiamo fatto"
% keyword: ABBIAMO fatto

\end{document}
